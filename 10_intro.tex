\subsection{Motivation}

Traditionally, in drug discovery, target identification to develop novel agents for any given disease is carried out on a case--by–-case basis. In this model, individual scientists act as project champions for targets based on available literature and local expertise. Nevertheless:

\begin{itemize}
    
    \item With increasing publication rates, it is becoming impossible to maintain an overview over an \textbf{increasingly vast scientific} literature. The large quantity not just biomedical research literature but also orthogonal perspectives continually being produced renders it impossible for individual researchers to keep up to date \cite{brown2018}.
    
    \begin{center}
    \emph{``Every two days we create as much information as we did from the dawn of civilization up until 2003, and the pace is increasing''}
    \end{center}
    \rightline{--- Eric Schmidt, former CEO of Google}
    
    \item In recent years a wealth of biological data from multiple domains has become available in \textbf{public data repositories} \cite{brown2018}. A firm integration of this extensive and extremely valuable corpus of scientific literature is necessary for a rapid generation of new, high-quality hypothesis for drug discovery without the need of human intervention \cite{ferrero2017, brown2018}.
    
    \item Drug discovery is \textbf{extremely costly and failure-prone} \cite{ferrero2017}. As the cost of successful drug development continues to increase, the productivity of the industry as a whole in launching new drugs remains flat \cite{brown2018}.
    
\end{itemize}

Hence, the putative target prioritisation in drug discovery:
\begin{itemize}
    \item Is of paramount importance to maximise the chances of success in clinic \cite{ferrero2017}.
    
    \item Ensures a sustainable business in the long term \cite{ferrero2017}.
\end{itemize}


\subsection{Proposed approach}
Based on the \href{https://goo.gl/QrfdfY}{initial proposal}, this PhD iCASE will focus on:
\begin{itemize}
    \item The development of an innovative integrated resource 
    \item which uses machine learning algorithms
    \item to exploit a newly generated semantic repository
    \item to facilitate a systematic genome scale ranking of potential targets in the areas of both type 2 diabetes and metabolic syndrome: a \textbf{putative therapeutic target priorisation} (PTTP)
\end{itemize}


The project will also explore different ranking systems to prioritize putative targets, for instance:


\begin{itemize}
    \item \textbf{Weighted averages} based on data types results.
    \item \textbf{Network topology} based scores.
    \item \textbf{Machine learning} approaches based on fingerprints derived from datatypes.
    \item \textbf{Enrichment} of the semantic repository with text mining approaches, linking to electronic medical records (EMR) or genomic resources available at the School of Medicine.
\end{itemize}

The different approaches for the target ranking can be evaluated by their ability to identify or prioritise a gold standard data set of targets provided by Exscientia.